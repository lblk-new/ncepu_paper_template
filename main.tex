%%%%%%%%%%%%%%%%%%%%%%%%%
%%%版权所有 华电北京数理系信计1702邓羊驼
%%%欢迎大家非商业使用 欢迎大家提出修改意见
%%%联系邮箱 dengyangtuo@gmail.com
%%%%%%%%%%%%%%%%%%%%%%%%%
\documentclass[oneside,10pt]{article} 
\usepackage{amsmath}
\usepackage[UTF8]{ctex}
\usepackage{titlesec}
\usepackage{titletoc}
\usepackage{lipsum}
\usepackage{amssymb}%数学字体
\usepackage{verbatim}
\usepackage{graphicx}
\usepackage{fancyhdr}
\usepackage[super]{cite}
\usepackage[a4paper,left=25mm,right=20mm,top=25mm,bottom=20mm]{geometry}%边距
\usepackage{fontspec}

\renewcommand\citeform[1]{[#1]}%把简单的cite引用改为上角标应用

\pagestyle{fancy}
%\fancyhf{}

\bibliographystyle{gbt7714-2005}%文献引用style,英文人名统一大写的,有一点怪,不过应该不是什么大问题



\newtheorem{th1}{定理}%定理环境名称:th1
\linespread{1.5}%1.5倍行距
\renewcommand\theequation{\arabic{section}-\arabic{equation}}

\renewcommand{\thetable}{\arabic{section}-\arabic{table}}
\renewcommand{\thefigure}{\arabic{section}-\arabic{figure}}



\newcommand{\cnabstractname}{摘要}
\newcommand{\enabstractname}{ABSTRACT}
\newenvironment{enabstract}{%
  \par\Large
  \noindent\mbox{}\hfill{\bfseries \enabstractname}\hfill\mbox{}\par
  \vskip 2.5ex}{\par\vskip 2.5ex}
\newenvironment{cnabstract}{%
  \par\Large
  \noindent\mbox{}\hfill{\bfseries \cnabstractname}\hfill\mbox{}\par
  \vskip 2.5ex}{\par\vskip 2.5ex}
\renewcommand{\cnabstractname}{\textbf{ {\LARGE 摘\quad 要}}}%中文摘要标题
 
\titleformat{\section}{\centering\LARGE\bfseries}{第 \thesection 章}{1em}{}
\titleformat{\subsection}{\raggedright\Large\bfseries}{\thesubsection}{1em}{}
\titleformat{\subsubsection}{\raggedright\large\bfseries}{\thesubsubsection}{1em}{}

%%%%%%%%%%%%%%%%%%%%%%%%%%%%%%%%%%%%%%%%%%%%%%%%%%%%
%以上基本不用改
%%%%%%%%%%%%%%%%%%%%%%%%%%%%%%%%%%%%%%%%%%%%%%%%%%%%

































































































%%%%%%%%%%%%%%%%%%%%%%%%%%%%%%%%%%%%%%%%%%%%%%%%%%%%
%以下是开头部分,基本不用改
%%%%%%%%%%%%%%%%%%%%%%%%%%%%%%%%%%%%%%%%%%%%%%%%%%%%
\begin{document}



%\fontsize{12pt}{5pt}
\large
%相关文献要求,正文小四,即12pt,即large
\fancyfoot[C]{\thepage}%摘要部分只有页脚没有页眉
\renewcommand{\headrulewidth}{0pt}%删掉页眉
\pagenumbering{Roman}%目录部分使用特殊页码





%%%%%%%%%%%%%%%%%%%%%%%%%%%%%%%%%%%%%%%%%%%%%%%%%%%%
%以下是中文摘要部分,修改
%%%%%%%%%%%%%%%%%%%%%%%%%%%%%%%%%%%%%%%%%%%%%%%%%%%%
\newpage
\addcontentsline{toc}{section}{\Large 摘 \quad 要}

\begin{cnabstract}
\large

%%%%%%%%%%%%%%%%%%%%%%%%%%%%%%%%%%%%%%%%%%%%%%%%%%%%
%此处应为正文
%%%%%%%%%%%%%%%%%%%%%%%%%%%%%%%%%%%%%%%%%%%%%%%%%%%%
1111中文摘要正文
\newline
\par\textbf{关键字: } 222qeqe,3333,4,55
%“\par在段首,表示另起一行,“\textbf{}”,花括号内的内容加粗显示
\end{cnabstract}




%%%%%%%%%%%%%%%%%%%%%%%%%%%%%%%%%%%%%%%%%%%%%%%%%%%%
%以下是英文摘要部分
%%%%%%%%%%%%%%%%%%%%%%%%%%%%%%%%%%%%%%%%%%%%%%%%%%%%
\newpage
\addcontentsline{toc}{section}{\Large ABSTRACT}
\begin{enabstract}
  \setmainfont{Times New Roman}
\large
11English abstract
%%%%%%%%%%%%%%%%%%%%%%%%%%%%%%%%%%%%%%%%%%%%%%%%%%%%
%此处应为英文正文
%%%%%%%%%%%%%%%%%%%%%%%%%%%%%%%%%%%%%%%%%%%%%%%%%%%%
\newline
\par\textbf{Keywords:} 22qeqe,33,444,51
%“\par在段首,表示另起一行,“\textbf{}”,花括号内的内容加粗显示
\end{enabstract}




%\fancyhf{}
%\tableofcontents%万一哪天脑抽了想生成目录呢?比如想照着pdf的目录抄一份word目录的页码,就注释开这行和上一行看看效果




%%%%%%%%%%%%%%%%%%%%%%%%%%%%%%%%%%%%%%%%%%%%%%%%%%%%
%以下是正文页眉页脚,不用改
%%%%%%%%%%%%%%%%%%%%%%%%%%%%%%%%%%%%%%%%%%%%%%%%%%%%
\newpage
\pagenumbering{arabic}
\pagestyle{fancy}
\fancyhf{}
\fancyhead[C]{华北电力大学本科毕业设计(论文)}
\fancyhead[L]{}
\fancyhead[R]{}
\fancyfoot[C]{\thepage}
\renewcommand{\headrulewidth}{0.4pt}



















%%%%%%%%%%%%%%%%%%%%%%%%%%%%%%%%%%%%%%%%%%%%%%%%%%%%
%%%%%%%%%%%%%%%%%%%%%%%%%%%%%%%%%%%%%%%%%%%%%%%%%%%%
%此处应为正文写作部分
%%%%%%%%%%%%%%%%%%%%%%%%%%%%%%%%%%%%%%%%%%%%%%%%%%%%
%%%%%%%%%%%%%%%%%%%%%%%%%%%%%%%%%%%%%%%%%%%%%%%%%%%%
\section{a}
\subsection{2}
\subsubsection{3}
\paragraph{4}
\subparagraph{5}
xx11xx22xx\cite{cal_pde,kingma2014adam,guo_convolutional_2016}
\begin{equation}
  \mathbb{I} = \int f(x)dx
\end{equation}
%%%%%%%%%%%%%%%%%%%%%%%%%%%%%%%%%%%%%%%%%%%%%%%%%%%%
%%%%%%%%%%%%%%%%%%%%%%%%%%%%%%%%%%%%%%%%%%%%%%%%%%%%
%此处应为正文写作部分
%%%%%%%%%%%%%%%%%%%%%%%%%%%%%%%%%%%%%%%%%%%%%%%%%%%%
%%%%%%%%%%%%%%%%%%%%%%%%%%%%%%%%%%%%%%%%%%%%%%%%%%%%











\newpage
\bibliography{ref}%填入bib文件名称
%\bibliography{ref0}








%%%%%%%%%%%%%%%%%%%%%%%%%%%%%%%%%%%%%%%%%%%%%%%%%%%%
%以下是附录写作部分,同正文格式基本相同
%%%%%%%%%%%%%%%%%%%%%%%%%%%%%%%%%%%%%%%%%%%%%%%%%%%%
\newpage
\appendix
\titleformat{\section}{\centering\LARGE\bfseries}{附录 \thesection}{1em}{}
\section{符号表}

\begin{center}
\begin{tabular}{l|l}
    \hline
    符号 & 意义\\ \hline
    $u$,$\overline{u}$  & 对应方程的解\\
    $\omega$ & 参数集合\\
    $\Delta t$&一个很短的时间间隔\\
    $\bigtriangledown$&求导向量算子\\
    $\bigtriangledown^2$& 拉普拉斯算子\\
    $V$,$U$&势函数\\
    $\varphi $,$\psi $,$\rho $& 方程前提条件中已知的某函数\\
    $Q$&最优化算法的目标叠加和\\
    $Q_i$&最优化算法的某个目标\\
    \hline
\end{tabular} 
\end{center}



\titleformat{\section}{\centering\LARGE\bfseries}{}{1em}{}

\newpage
\section{致谢}

%%%%%%%%%%%%%%%%%%%%%%%%%%%%%%%%%%%%%%%%%%%%%%%%%%%%
%%%%%%%%%%%%%%%%%%%%%%%%%%%%%%%%%%%%%%%%%%%%%%%%%%%%
%此处应为致谢正文部分
%%%%%%%%%%%%%%%%%%%%%%%%%%%%%%%%%%%%%%%%%%%%%%%%%%%%
%%%%%%%%%%%%%%%%%%%%%%%%%%%%%%%%%%%%%%%%%%%%%%%%%%%%

%???????
%???????\\











\end{document}

